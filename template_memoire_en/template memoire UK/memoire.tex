\documentclass[11pt,a4paper,oneside]{book}
\usepackage[hmargin={1.25in,1.25in},vmargin={1.25in,1.25in}]{geometry}

\makeindex
\usepackage{textcomp}
\usepackage{fancyhdr}
\usepackage{makeidx}
\pagestyle{myheadings}
\fancyhf{}
\rhead[\leftmark]{thepage}

\usepackage[latin1]{inputenc}
\usepackage{url}

\parindent0em
\parskip1.5ex

\begin{document}

\frontmatter
\begin{titlepage}
\begin{center}
\textbf{UNIVERSIT\'E LIBRE DE BRUXELLES}\\
\textbf{Facult\'e des Sciences}\\
\textbf{D\'epartement d'Informatique}
\vfill{}\vfill{}

{\Huge  Development of an automatically configurable ant colony optimization framework}

{\Huge \par}
\begin{center}{\LARGE Aldar Saranov}\end{center}{\Huge \par}
\vfill{}\vfill{}
\begin{flushright}{\large \textbf{Promotor :} Prof. Thomas St{\"u}tzle}\hfill{}{\large Master Thesis in Computer Sciences}\\
{\large }\hfill{}{}\end{flushright}{\large\par}
\vfill{}\vfill{}\enlargethispage{3cm}
\textbf{Academic year 2016~-~2017+1}
\end{center}
\end{titlepage}
\newpage
\thispagestyle{empty} 
\null

\newenvironment{vcenterpage}
{\newpage\thispagestyle{empty} 
\vspace*{\fill}}
{\vspace*{\fill}\par\pagebreak}

\begin{vcenterpage}
\begin{flushright}
    \large\em\null\vskip1in 
    Dedicated to my mother Lena, who\\
   was always sincerely supporting me\vfill
  \end{flushright}
\end{vcenterpage}
\thispagestyle{empty}
\vspace*{5cm}

\begin{quotation}
\noindent ``\emph{If one does not know to which port one is sailing, no wind is favorable.}''
\begin{flushright}\textbf{Lucius Annaeus Seneca, 1st century AD}\end{flushright}
\end{quotation}

\medskip

\begin{quotation}
\noindent ``\emph{The general, unable to control his irritation, will launch his men
to the assault like swarming ants, with the result that one-third of
his men are slain, while the town still remains untaken. Such are
the disastrous effects of a siege.}''
\begin{flushright}\textbf{Sun Tzu, "The Art of War", 5th century BC}\end{flushright}
\end{quotation}
\chapter*{Acknowledgment}
\thispagestyle{empty} 

\noindent I want to thank ...

\thispagestyle{empty} 
\setcounter{page}{0}
\tableofcontents
\mainmatter 
\chapter{Introduction}
\setcounter{page}{1}

\vspace*{0.5cm}

\section{Background and objectives of the thesis}

Blabla

For a quick and detailed introduction to  \LaTeX et \LaTeX2e, you can consult, in addition to classical books, \cite{lamp,mittel} and the website \cite{oetik}.

You may need to use other packages than those mentioned at the beginning of the file! To write algorithms, see for example the site \cite{fiorio}.

\section{Section name}

Blabla

Blabla

\section{Structure of the thesis}

In chapter \ref{chap2} ...

Blabla

\section{Contributions}

Our main contributions are ...\vspace{1cm}

\clearpage
\section{Notations}

It may be interesting to gather here the main notations, their meaning, and the page where it is defined in more detail or possibly a bibliographic reference.

\chapter{Chapter name}

It is important to always add all the needed bibliographic references~\cite{ref1,ref3}.

\section{Section name}

When an important notion is defined in a large document, it may be useful to add it in an index\index{Important notion}, this will allow the reader to quickly find this definition if he has forgotten it.

\subsection{Sub-section name}

This provides a more structured text.

\paragraph{Title of a paragraph}

You can also have subsubsections, but it is usually better to avoid it (in order to prevent to have  sections numbering such as 3.2.1.5).

Similarly, we must choose how to number the definitions, lemmas, proposals and theorems. We may manage the corresponding counters to operate at the global level, at the level of sections, ... and we may prefer for example to avoid to have a theorem 1 which follows a proposal 15.

\section{Second section}

\section{...}Blabla~\cite{ref2}.

\chapter{Second chapter}

\label{chap2}

\section{...}Blabla~\cite{ref4}.

\section{...}
\label{sec-untel}

\chapter{Next one}

\section{...}

\section{...}

\chapter*{Conclusions}

The conclusions are to be written with care\index{Care}, because it will be sometimes the part that could convince a potential reader to read the whole document.

\appendix

\backmatter

\printindex % use makeindex to generate the index

\bibliographystyle{plain}

\bibliography{biblio} %use bibtex to generate the bibliography

\end{document}